% This contents of this file will be inserted into the _Solutions version of the
% output tex document.  Here's an example:

% If assignment with subquestion (1.a) requires a written response, you will
% find the following flag within this document: <SCPD_SUBMISSION_TAG>_1a
% In this example, you would insert the LaTeX for your solution to (1.a) between
% the <SCPD_SUBMISSION_TAG>_1a flags.  If you also constrain your answer between the
% START_CODE_HERE and END_CODE_HERE flags, your LaTeX will be styled as a
% solution within the final document.

% Please do not use the '<SCPD_SUBMISSION_TAG>' character anywhere within your code.  As expected,
% that will confuse the regular expressions we use to identify your solution.
\def\assignmentnum{3 }
\def\assignmenttitle{XCS236 Problem Set \assignmentnum}

\documentclass{article}
\usepackage[top = 1.0in]{geometry}

\usepackage{graphicx}

\usepackage[utf8]{inputenc}
\usepackage{listings}
\usepackage[dvipsnames]{xcolor}
\usepackage{bm}
\usepackage{algorithm}
\usepackage{algpseudocode}
\usepackage{framed}

\definecolor{codegreen}{rgb}{0,0.6,0}
\definecolor{codegray}{rgb}{0.5,0.5,0.5}
\definecolor{codepurple}{rgb}{0.58,0,0.82}
\definecolor{backcolour}{rgb}{0.95,0.95,0.92}

\lstdefinestyle{mystyle}{
    backgroundcolor=\color{backcolour},   
    commentstyle=\color{codegreen},
    keywordstyle=\color{magenta},
    stringstyle=\color{codepurple},
    basicstyle=\ttfamily\footnotesize,
    breakatwhitespace=false,         
    breaklines=true,                 
    captionpos=b,                    
    keepspaces=true,                 
    numbersep=5pt,                  
    showspaces=false,                
    showstringspaces=false,
    showtabs=false,                  
    tabsize=2
}

\lstset{style=mystyle}

\newcommand{\di}{{d}}
\newcommand{\nexp}{{n}}
\newcommand{\nf}{{p}}
\newcommand{\vcd}{{\textbf{D}}}
\newcommand{\Int}{\mathbb{Z}}
\newcommand\bb{\ensuremath{\mathbf{b}}}
\newcommand\bs{\ensuremath{\mathbf{s}}}
\newcommand\bp{\ensuremath{\mathbf{p}}}
\newcommand{\relu} { \operatorname{ReLU} }
\newcommand{\smx} { \operatorname{softmax} }
\newcommand\bx{\ensuremath{\mathbf{x}}}
\newcommand\bz{\ensuremath{\mathbf{z}}}
\newcommand\bh{\ensuremath{\mathbf{h}}}
\newcommand\bc{\ensuremath{\mathbf{c}}}
\newcommand\bW{\ensuremath{\mathbf{W}}}
\newcommand\by{\ensuremath{\mathbf{y}}}
\newcommand\bo{\ensuremath{\mathbf{o}}}
\newcommand\be{\ensuremath{\mathbf{e}}}
\newcommand\ba{\ensuremath{\mathbf{a}}}
\newcommand\bu{\ensuremath{\mathbf{u}}}
\newcommand\bv{\ensuremath{\mathbf{v}}}
\newcommand\bP{\ensuremath{\mathbf{P}}}
\newcommand\bg{\ensuremath{\mathbf{g}}}
\newcommand\bX{\ensuremath{\mathbf{X}}}
% real numbers R symbol
\newcommand{\Real}{\mathbb{R}}

% encoder hidden
\newcommand{\henc}{\bh^{\text{enc}}}
\newcommand{\hencfw}[1]{\overrightarrow{\henc_{#1}}}
\newcommand{\hencbw}[1]{\overleftarrow{\henc_{#1}}}

% encoder cell
\newcommand{\cenc}{\bc^{\text{enc}}}
\newcommand{\cencfw}[1]{\overrightarrow{\cenc_{#1}}}
\newcommand{\cencbw}[1]{\overleftarrow{\cenc_{#1}}}

% decoder hidden
\newcommand{\hdec}{\bh^{\text{dec}}}

% decoder cell
\newcommand{\cdec}{\bc^{\text{dec}}}

\usepackage[hyperfootnotes=false]{hyperref}
\hypersetup{
  colorlinks=true,
  linkcolor = blue,
  urlcolor  = blue,
  citecolor = blue,
  anchorcolor = blue,
  pdfborderstyle={/S/U/W 1}
}
\usepackage{nccmath}
\usepackage{mathtools}
\usepackage{graphicx,caption}
\usepackage[shortlabels]{enumitem}
\usepackage{epstopdf,subcaption}
\usepackage{psfrag}
\usepackage{amsmath,amssymb,epsf}
\usepackage{verbatim}
\usepackage{cancel}
\usepackage{color,soul}
\usepackage{bbm}
\usepackage{listings}
\usepackage{setspace}
\usepackage{float}
\definecolor{Code}{rgb}{0,0,0}
\definecolor{Decorators}{rgb}{0.5,0.5,0.5}
\definecolor{Numbers}{rgb}{0.5,0,0}
\definecolor{MatchingBrackets}{rgb}{0.25,0.5,0.5}
\definecolor{Keywords}{rgb}{0,0,1}
\definecolor{self}{rgb}{0,0,0}
\definecolor{Strings}{rgb}{0,0.63,0}
\definecolor{Comments}{rgb}{0,0.63,1}
\definecolor{Backquotes}{rgb}{0,0,0}
\definecolor{Classname}{rgb}{0,0,0}
\definecolor{FunctionName}{rgb}{0,0,0}
\definecolor{Operators}{rgb}{0,0,0}
\definecolor{Background}{rgb}{0.98,0.98,0.98}
\lstdefinelanguage{Python}{
    numbers=left,
    numberstyle=\footnotesize,
    numbersep=1em,
    xleftmargin=1em,
    framextopmargin=2em,
    framexbottommargin=2em,
    showspaces=false,
    showtabs=false,
    showstringspaces=false,
    frame=l,
    tabsize=4,
    % Basic
    basicstyle=\ttfamily\footnotesize\setstretch{1},
    backgroundcolor=\color{Background},
    % Comments
    commentstyle=\color{Comments}\slshape,
    % Strings
    stringstyle=\color{Strings},
    morecomment=[s][\color{Strings}]{"""}{"""},
    morecomment=[s][\color{Strings}]{'''}{'''},
    % keywords
    morekeywords={import,from,class,def,for,while,if,is,in,elif,else,not,and,or
    ,print,break,continue,return,True,False,None,access,as,,del,except,exec
    ,finally,global,import,lambda,pass,print,raise,try,assert},
    keywordstyle={\color{Keywords}\bfseries},
    % additional keywords
    morekeywords={[2]@invariant},
    keywordstyle={[2]\color{Decorators}\slshape},
    emph={self},
    emphstyle={\color{self}\slshape},
%
}
\lstMakeShortInline|

\pagestyle{empty} \addtolength{\textwidth}{1.0in}
\addtolength{\textheight}{0.5in}
\addtolength{\oddsidemargin}{-0.5in}
\addtolength{\evensidemargin}{-0.5in}
\newcommand{\ruleskip}{\bigskip\hrule\bigskip}
\newcommand{\nodify}[1]{{\sc #1}}
\newenvironment{answer}{\sf \begingroup\color{ForestGreen}}{\endgroup}%

\setlist[itemize]{itemsep=2pt, topsep=0pt}
\setlist[enumerate]{itemsep=6pt, topsep=0pt}

\setlength{\parindent}{0pt}
\setlength{\parskip}{4pt}
\setlist[enumerate]{parsep=4pt}
\setlength{\unitlength}{1cm}

\renewcommand{\Re}{{\mathbb R}}
\newcommand{\R}{\mathbb{R}}
\newcommand{\what}[1]{\widehat{#1}}

\renewcommand{\comment}[1]{}
\newcommand{\mc}[1]{\mathcal{#1}}
\newcommand{\half}{\frac{1}{2}}

\DeclareMathOperator*{\argmin}{arg\,min}
\DeclareMathOperator*{\argmax}{arg\,max}

\def\KL{D_{\text{KL}}}
\def\xsi{x^{(i)}}
\def\ysi{y^{(i)}}
\def\zsi{z^{(i)}}
\def\E{\mathbb{E}}
\def\calN{\mathcal{N}}
\def\calX{\mathcal{X}}
\def\calY{\mathcal{Y}}
\def\calZ{\mathcal{Z}}
\def\calD{\mathcal{D}}
\def\calL{\mathcal{L}}
\def\slack{\url{http://xcs236-scpd.slack.com/}}
\def\zipscriptalt{\texttt{python zip\_submission.py}}
\DeclarePairedDelimiter\abs{\lvert}{\rvert}%
 
\usepackage{bbding}
\usepackage{pifont}
\usepackage{wasysym}
\usepackage{amssymb}
\usepackage{framed}
\usepackage{scrextend}

\newcommand{\alns}[1] {
	\begin{align*} #1 \end{align*}
}

\newcommand{\pd}[2] {
 \frac{\partial #1}{\partial #2}
}
\renewcommand{\Re} { \mathbb{R} }
\newcommand{\btx} { \mathbf{\tilde{x}} }
\newcommand{\bth} { \mathbf{\tilde{h}} }
\newcommand{\sigmoid} { \operatorname{\sigma} }
\newcommand{\CE} { \operatorname{CE} }
\newcommand{\byt} { \hat{\by} }
\newcommand{\yt} { \hat{y} }

\newcommand{\oft}[1]{^{(#1)}}
\newcommand{\fone}{\ensuremath{F_1}}

\newcommand{\ac}[1]{ {\color{red} \textbf{AC:} #1} }
\newcommand{\ner}[1]{\textbf{\color{blue} #1}}

\newcommand{\dataset}{\mathcal{D}}
\newcommand{\task}{\mathcal{T}}
\newcommand{\supportdata}{\mathcal{D}^\mathrm{tr}}
\newcommand{\querydata}{\mathcal{D}^\mathrm{ts}}
\newcommand{\support}[1]{{#1}^\mathrm{tr}}
\newcommand{\query}[1]{{#1}^\mathrm{ts}}
\begin{document}
\pagestyle{myheadings} \markboth{}{\assignmenttitle}

% <SCPD_SUBMISSION_TAG>_entire_submission

This handout includes space for every question that requires a written response.
Please feel free to use it to handwrite your solutions (legibly, please).  If
you choose to typeset your solutions, the |README.md| for this assignment includes
instructions to regenerate this handout with your typeset \LaTeX{} solutions.
\ruleskip

\LARGE
2a
\normalsize
% <SCPD_SUBMISSION_TAG>_2a
\begin{answer}
    To help you start the proof:
    Using the chain rule and the fact that $\sigma^{\prime}(x) = \sigma(x)(1-\sigma(x))$,

    \begin{center}
        $\frac{\partial L_{G}^{\text{minimax}}}{\partial \theta} = \E_{\bm{z} \sim \calN(0,I)}\left[- \frac{\sigma^{\prime}\left(h_{\phi}\left(G_{\theta}\left(\bm{z}\right)\right)\right)}{1 - \sigma\left(h_{\phi}\left(G_{\theta}\left(\bm{z}\right)\right)\right)} \frac{\partial}{\partial \theta} h_{\phi}\left(G_{\theta}\left(\bm{z}\right)\right)\right] = $ \\
    \end{center}

    % ### START CODE HERE ###
    % ### END CODE HERE ###
\end{answer}
% <SCPD_SUBMISSION_TAG>_2a

\clearpage

\LARGE
3a
\normalsize
% <SCPD_SUBMISSION_TAG>_3a
\begin{answer}
    To help you get started with the proof:
    If we break the expectation up, we see that

    \begin{center}
        $L_{D}(\phi; \theta) = - \E_{\bm{x} \sim p_{\text{data}}(\bm{x})}[\log D_{\phi}(\bm{x})] - \E_{\bm{x} \sim p_{\theta}(\bm{x})}[\log(1-D_{\phi}(\bm{x}))]$ \\ 
        $ = - \int p_{\text{data}}(\bm{x}) \log D_{\phi}(\bm{x})d\bm{x} - \int p_{\theta}(\bm{x}) \log(1-D_{\phi}(\bm{x}))d\bm{x}$ \\
        $ = - \int \left(p_{\text{data}}(\bm{x}) \log D_{\phi}(\bm{x}) + p_{\theta}(\bm{x}) \log(1-D_{\phi}(\bm{x})) \right)d\bm{x}$ \\
        $ = \int f(D_{\phi}(\bm{x}))d\bm{x}$
    \end{center}

    We can set $L^{\prime}_D(\phi; \theta) = 0$ to obtain the optimal $L^{\prime}_D$. This yields 
    
    \begin{center}
        $L^{\prime}_D(\phi;\theta) = \frac{d}{d D_{\phi}(\bm{x})} \int f(D_{\phi}(\bm{x}))d\bm{x} = \int \frac{d}{d D_{\phi}(\bm{x})} f(D_{\phi}(\bm{x}))d\bm{x} = 0$
    \end{center}
    
    Now try to apply the hint!
    % ### START CODE HERE ###
    % ### END CODE HERE ###
\end{answer}
% <SCPD_SUBMISSION_TAG>_3a

\LARGE
3b
\normalsize
% <SCPD_SUBMISSION_TAG>_3b
\begin{answer}
    To help you get started, note that

    \begin{center}
        $D_{\phi}(\bm{x}) = \sigma(h_{\phi}(\bm{x})) = \frac{1}{1 + e^{-h_{\phi}(\bm{x})}}$
    \end{center}

    Setting this to the expression for $D^{*}(\bm{x})$ in part 3a solution, we find that

    % ### START CODE HERE ###
    % ### END CODE HERE ###
\end{answer}
% <SCPD_SUBMISSION_TAG>_3b

\LARGE
3c
\normalsize
% <SCPD_SUBMISSION_TAG>_3c
\begin{answer}
    To get started
    \begin{center}
        $L_{G}(\theta; \phi) = \E_{p_{\theta}(\bm{x})}[\log (1-D_{\phi}(\bm{x}))] - \E_{p_{\theta}(\bm{x})}[\log D_{\phi}(\bm{x})]$ \\
        $ = \E_{p_{\theta}(\bm{x})} \left[\log \frac{1-D_{\phi}(\bm{x})}{D_{\phi}(\bm{x})}\right]$ \\
    \end{center}

    % ### START CODE HERE ###
    % ### END CODE HERE ###
\end{answer}
% <SCPD_SUBMISSION_TAG>_3c

\LARGE
3d
\normalsize
% <SCPD_SUBMISSION_TAG>_3d
\begin{answer}
    % ### START CODE HERE ###
    % ### END CODE HERE ###
\end{answer}
% <SCPD_SUBMISSION_TAG>_3d

\clearpage

\LARGE
4a
\normalsize
% <SCPD_SUBMISSION_TAG>_4a
\begin{answer}
    To help you get started:
    
    \begin{center}
        $h_{\phi}(x,y) = \log \frac{p_{\text{data}}(\bm{x},y)}{p_{\theta}(\bm{x}, y)}$ \\
        $ = \log \frac{p_{\text{data}}(\bm{x} \mid y)}{p_{\theta}(\bm{x} \mid y)} + \log \frac{p_{\text{data}}(y)}{p_{\theta}(y)}$ \\
        $ = \log \frac{p_{\text{data}}(\bm{x} \mid y)}{p_{\theta}(\bm{x} \mid y)} = $ \\
    \end{center}

    % ### START CODE HERE ###
    % ### END CODE HERE ###
\end{answer}
% <SCPD_SUBMISSION_TAG>_4a

\clearpage

\LARGE
5a
\normalsize
% <SCPD_SUBMISSION_TAG>_5a
\begin{answer}
    To help you get started: 
    \begin{center}
        $\text{KL}(p_{\theta}(x) \mid\mid p_{\text{data}}(x)) = \E_{x \sim \calN(\theta, \epsilon^2)}\left[ \log \frac{\exp(- \frac{1}{2\epsilon^2}(x-\theta)^2)}{\exp(- \frac{1}{2\epsilon^2}(x - \theta_0)^2)}\right] = $ \\
    \end{center}

    % ### START CODE HERE ###
    % ### END CODE HERE ###
\end{answer}
% <SCPD_SUBMISSION_TAG>_5a


\LARGE
5b
\normalsize
% <SCPD_SUBMISSION_TAG>_5b
\begin{answer}
    % ### START CODE HERE ###
    % ### END CODE HERE ###
\end{answer}
% <SCPD_SUBMISSION_TAG>_5b

\LARGE
5c
\normalsize
% <SCPD_SUBMISSION_TAG>_5c
\begin{answer}
    % ### START CODE HERE ###
    % ### END CODE HERE ###
\end{answer}
% <SCPD_SUBMISSION_TAG>_5c

\LARGE
5d
\normalsize
% <SCPD_SUBMISSION_TAG>_5d
\begin{answer}
    % ### START CODE HERE ###
    % ### END CODE HERE ###
\end{answer}
% <SCPD_SUBMISSION_TAG>_5d

% <SCPD_SUBMISSION_TAG>_entire_submission

\end{document}